\begin{frame}
    \frametitle{Giới thiệu Paper}

    \textbf{Paper:} \textit{Energy management and power quality improvement of microgrid system through modified water wave optimization} (Energy Reports, 2023) \cite{choudhury2023energy}

    \vspace{0.5em}
    \textbf{Mục tiêu:}
    \begin{itemize}
        \item Quản lý năng lượng (EMS) và cải thiện chất lượng điện năng (PQ) trong microgrid (MG) kết hợp: \textbf{PEM Fuel Cell}, \textbf{Pin Lithium-ion}, \textbf{Supercapacitor}.
        \item Tối ưu tham số bộ điều khiển PI để:
        \begin{itemize}
            \item Nâng cao hiệu suất
            \item Giảm tiêu thụ nhiên liệu
            \item Tăng độ ổn định hệ thống
        \end{itemize}
    \end{itemize}

    \vspace{0.5em}
    \textbf{Phương pháp:}
    \begin{itemize}
        \item Áp dụng \textbf{Water Wave Optimization (WWO)} để tự động tinh chỉnh tham số PI, đáp ứng tải nhanh, duy trì PQ ổn định
        \item Ưu điểm WWO: đơn giản, ít tham số, hiệu quả với quần thể nhỏ
    \end{itemize}
\end{frame}

\begin{frame}
    \frametitle{Bài toán Tối ưu}

    \textbf{Mục tiêu:} 
    \begin{itemize}
        \item Tối ưu \textbf{6 hệ số PI} cho 3 nguồn: Fuel Cell (FC), Battery (Bat), Supercapacitor (SC)
        \item Giảm \textbf{tổng tích phân bình phương sai số} điều khiển
    \end{itemize}

    \vspace{0.5em}
    \textbf{Biến quyết định:}
    \[
    \mathbf{X} = [k_{p,FC}, k_{i,FC}, k_{p,Bat}, k_{i,Bat}, k_{p,SC}, k_{i,SC}]
    \]
    \textbf{Hàm mục tiêu:}
    \begin{align}
    J_{\text{total}} 
    &= w_1 \int_{0}^{T} t \cdot \left| P_{\text{FC}}^{\text{ref}}(t) - P_{\text{FC}}(t) \right| dt \nonumber \&\quad + w_2 \int_{0}^{T} t \cdot \left| SOC_{\text{Bat}}^{\text{ref}}(t) - SOC_{\text{Bat}}(t) \right| dt \nonumber \\
    &\quad + w_3 \int_{0}^{T} t \cdot \left| SOC_{\text{SC}}^{\text{ref}}(t) - SOC_{\text{SC}}(t) \right| dt
    \end{align}
    \textbf{Ràng buộc:} giới hạn $k_p,k_i$ ($0 \le k_p \le 10,\; 0 \le k_i \le 1$), SOC an toàn, công suất/duty cycle trong \([0,1]\).
    
\end{frame}



\begin{frame}
    \frametitle{WWO trong bài báo}

    \textbf{WWO gốc:}
    \begin{itemize}
        \item Khởi tạo quần thể sóng $P$, bước sóng $\lambda$, chiều cao sóng $h_{\text{max}}$, hệ số $\alpha$, $\beta$, $k_{\text{max}}$.
        \item Đánh giá hàm mục tiêu $\to$ tìm nghiệm tốt nhất $x^*$.
        \item \textbf{Vòng lặp:} 
        \begin{itemize}
            \item \textit{Propagation:} tạo nghiệm mới $x'$ quanh $x$.
            \item Nếu $f(x') > f(x)$: thay thế, có thể thực hiện \textit{Breaking} nếu tốt hơn $x^*$.
            \item Nếu không: giảm $h$, nếu $h=0$ thì \textit{Refraction} (dịch về gần $x^*$).
        \end{itemize}
        \item Cập nhật $\lambda$ và lặp lại tới khi dừng.
    \end{itemize}

    \textbf{Nhược điểm:}
    \begin{itemize}
        \item Mạnh khai thác cục bộ nhưng yếu khám phá toàn cục, dễ mất đa dạng và hội tụ sớm.
        \item Quần thể cố định, khó cân bằng giữa exploration \& exploitation.
    \end{itemize}
    => MWWO thay đổi quy mô quần thể và nhịp bước sóng theo tiến trình tối ưu.

\end{frame}

\begin{frame}{Cải tiến chính của MWWO}
    \textbf{1) Hệ số ``bước sóng'' thích nghi (adaptive wavelength coefficient)}
    \begin{itemize}
        \item \textbf{Vấn đề của WWO gốc:} quy tắc cập nhật bước sóng $\lambda$ dựa vào hệ số $\alpha$ cố định. Khi $\alpha$ không đổi, mức ``sải bước'' ở pha dò tìm có thể không còn phù hợp theo thời gian, dễ mất đa dạng hoặc quá thô khi cần tinh chỉnh.
        \item \textbf{Cách MWWO sửa:} thay $\alpha$ cố định bằng $\alpha$ thích nghi theo thế hệ:
        \begin{equation}
            \alpha_i = \alpha_{\max} \left( \frac{i_{\text{MGN}} - i_{\text{CGN}} + 1}{i_{\text{MGN}}} \right)^{\gamma}
        \end{equation}

        
        \begin{itemize}
            \item $i_{\text{CGN}}$: chỉ số vòng lặp hiện tại (Current Generation Number).
            \item $i_{\text{MGN}}$: tổng số vòng lặp tối đa của thuật toán (Maximum Generation Number).
            \item $\gamma$: tham số điều chỉnh tốc độ giảm của $\alpha$.
        \end{itemize}
        \item Cơ chế này giúp dò tìm mạnh ở đầu (exploration) rồi thu hẹp sải bước để khai thác (exploitation) về cuối, giảm nguy cơ kẹt cục bộ trên các hàm đa cực trị.
        \item Trong thuật toán MWWO, việc cập nhật $\alpha \rightarrow \alpha_i$ được đặt thành bước riêng (Step 4) trước khi lan truyền (propagation).
    \end{itemize}
    
\end{frame}

\begin{frame}
    \frametitle{Cải tiến chính của MWWO}

    \textbf{2) Kích thước quần thể thích nghi (adaptive population size)}
    \begin{itemize}
        \item \textbf{Vấn đề của WWO gốc:} giữ kích thước quần thể cố định qua mọi thế hệ $\rightarrow$ kém linh hoạt, dễ kẹt cực trị cục bộ (premature convergence) khi phải cân bằng dò tìm rộng (exploration) và khai thác sâu (exploitation).
        \item \textbf{Cách MWWO sửa:} dùng một quy luật tuyến tính giảm dần quy mô quần thể theo số thế hệ:
        \begin{equation}
        n = n_{\max} - (n_{\max} - n_{\min}) \cdot \frac{i_{\text{CGN}}}{i_{\text{MGN}}} \quad
        \end{equation}
        \item \textbf{Ý nghĩa:} 
        \begin{enumerate}
            \item Đầu quá trình $\rightarrow$ quần thể lớn để quét rộng không gian nghiệm.
            \item Càng về sau $\rightarrow$ quần thể nhỏ để tập trung tinh chỉnh cục bộ.
        \end{enumerate}
        \item \textbf{Lợi ích:} (i) cân bằng exploration/exploitation, giảm kẹt cục bộ; (ii) mỗi lần thu nhỏ quần thể thì giảm số lần đánh giá fitness, hạ chi phí tính toán mà vẫn cho nghiệm tốt hơn.
    \end{itemize}

    \vspace{0.5em}
    
\end{frame}

\begin{frame}
    \frametitle{Kết quả so sánh PI -- WWO -- MWWO}

    \begin{table}[h]
        \centering
        \begin{tabular}{lccc}
            \hline
            \textbf{Tiêu chí} & \textbf{PI} & \textbf{WWO} & \textbf{MWWO} \\
            \hline
            Thời gian thực thi (s)     & 120   & 59    & \textbf{48}    \\
            Tiêu thụ H$_2$ (g/1000s)   & 50.50 & 41.39 & \textbf{40.81} \\
            Công suất tải FC (W)       & 1349  & 1580  & \textbf{2300}  \\
            Hiệu suất tổng (\%)        & 69.32 & 81.51 & \textbf{89.67} \\
            Stress FC                  & 22.27 & 17.59 & \textbf{12.54} \\
            SoC pin (\%)               & 70--52 & 70--55 & \textbf{70--59} \\
            \hline
        \end{tabular}
    \end{table}

    \vspace{0.5em}
    \textbf{Quan sát:}
    \begin{itemize}
        \item MWWO vượt trội ở hầu hết tiêu chí: hiệu suất cao nhất, tiết kiệm nhiên liệu hơn, giảm stress thiết bị.
        \item Cải thiện rõ rệt chất lượng điện năng (PQ) và duy trì SoC pin ở mức an toàn hơn.
    \end{itemize}
\end{frame}
