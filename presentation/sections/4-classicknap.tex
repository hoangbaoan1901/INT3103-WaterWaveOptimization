\begin{frame}
    \frametitle{Áp dụng WWO trong classic knapsack}
    \begin{block}{Phát biểu bài toán:}
        Cho $n$ đồ vật được đánh số từ $1$ đến $n$, trong đó đồ vật thứ $i$ có khối lượng $w_i$ và giá trị $v_i$. 
        Yêu cầu chọn một tập các đồ vật sao cho tổng khối lượng không vượt quá giới hạn $W$ và tổng giá trị đạt lớn nhất có thể. 
    \end{block}

    \textbf{Nhược điểm của WWO gốc:}
        \begin{itemize}
            \item \textbf{Propagation ngẫu nhiên, không định hướng:} dễ bỏ item giá trị cao hoặc thêm item chất lượng kém, gây nhiễu và chậm hội tụ.
            \item \textbf{Local search đơn giản:} thiếu các hoán đổi nhiều item $\rightarrow$ dễ kẹt ở nghiệm cục bộ.
            \item \textbf{Xử lý nghiệm vi phạm kém:} dùng hàm phạt, nhiều bước lãng phí cho nghiệm infeasible.
            \item \textbf{Thiếu cơ chế đa dạng hoá:} không có chiến lược escape khi hội tụ sớm.
            \item \textbf{Không tận dụng đặc thù knapsack:} bỏ qua thông tin mật độ giá trị (density) để định hướng chọn/bỏ item.
        \end{itemize}
\end{frame}

\begin{frame}
    \frametitle{Cải tiến 1: Propagation có định hướng theo điểm số}
    \textbf{Lý do cần cải tiến:}
    \begin{itemize}
        \item WWO gốc chỉ \textit{lật ngẫu nhiên k bit} quanh nghiệm hiện tại $\rightarrow$ dễ bỏ item tốt hoặc thêm item tệ $\rightarrow$ lãng phí bước lan truyền.
        \item Knapsack: density là chỉ số mạnh để chọn/bỏ đúng.
    \end{itemize}
    \textbf{Mô tả:}
    \begin{itemize}
        \item Cải tiến: tính score cho từng item:
        \begin{itemize}
            \item Item chưa chọn: score cao nếu density ($v/w$) lớn + thêm yếu tố giá trị.
            \item Item đang chọn: score dựa trên nghịch đảo density (density thấp $\rightarrow$ score cao).
        \end{itemize}
         \item Chọn \texttt{max\_candidates} item có score cao nhất bằng \textbf{heap} (tìm top-k hiệu quả).
        \item \textbf{Ngẫu nhiên hóa nhẹ} thứ tự các ứng viên để tránh cứng nhắc và duy trì đa dạng.
        \item Lật \texttt{num\_changes} item từ danh sách ứng viên.
        \item Nếu nghiệm vi phạm ràng buộc $\rightarrow$ gọi \texttt{repair\_and\_fill\_solution}.
   
    \end{itemize}
    \textbf{Khắc phục nhược điểm:}
    \begin{itemize}
        \item Giảm nhiễu ngẫu nhiên nhưng vẫn giữ một phần ngẫu nhiên để tránh kẹt local optima.
        \item Tăng xác suất cải thiện $\rightarrow$ hội tụ nhanh, ổn định hơn.
    \end{itemize}
\end{frame}


\begin{frame}
    \frametitle{Cải tiến 2: Local search giàu toán tử (1–1, 2–1, 1–2, 2–2)}
    \textbf{Lý do cần cải tiến:}
    \begin{itemize}
        \item LS gốc của WWO đơn giản, dễ kẹt ở local optima.
        \item Knapsack đôi khi cần bỏ nhiều item nhỏ để thêm item lớn (hoặc ngược lại).
    \end{itemize}
    \textbf{Mô tả:}
    \begin{itemize}
        \item Chọn \textbf{top-density unselected} (item chưa chọn tốt nhất) và \textbf{worst-density selected} (item đang chọn tệ nhất).
        \item Thử các hoán đổi:
        \begin{itemize}
            \item 1–1: thay 1 xấu bằng 1 tốt.
            \item 2–1: bỏ 2 xấu, thêm 1 tốt.
            \item 1–2: bỏ 1 xấu, thêm 2 tốt.
            \item 2–2: bỏ 2 xấu, thêm 2 tốt.
        \end{itemize}
    \end{itemize}

    \textbf{Khắc phục nhược điểm:}
    \begin{itemize}
        \item Cho phép nhảy qua rào cản local optima.
        \item Khám phá không gian nghiệm rộng hơn, vẫn dựa trên density.
    \end{itemize}
\end{frame}


\begin{frame}
    \frametitle{Cải tiến 3: Repair-then-fill khi quá tải}
    \textbf{Lý do cần cải tiến:}
    \begin{itemize}
        \item WWO gốc cho MKP dùng hàm phạt: nghiệm infeasible vẫn giữ $\rightarrow$ lãng phí bước tìm kiếm.
        \item 1D knapsack: repair đơn giản và nhanh.
    \end{itemize}
    \textbf{Mô tả:}
    \begin{itemize}
        \item Nếu nghiệm vượt capacity:
        \begin{itemize}
            \item Bỏ item tệ nhất (density thấp nhất) cho đến khi feasible.
            \item Lấp đầy bằng item density cao nhất còn lại mà vẫn vừa capacity.
        \end{itemize}
    \end{itemize}
    \textbf{Khắc phục nhược điểm:}
    \begin{itemize}
        \item Đảm bảo nghiệm feasible sớm.
        \item Fitness luôn phản ánh giá trị thật, không ảo do phạt.
        \item Giảm thời gian “chữa cháy” trong quá trình chạy.
    \end{itemize}
\end{frame}

\begin{frame}
    \frametitle{Cải tiến 4: Đa dạng hoá khi trì trệ}
    \textbf{Lý do cần cải tiến:}
    \begin{itemize}
        \item WWO gốc dễ hội tụ sớm nếu quần thể bị hút vào 1 vùng nghiệm.
        \item Landscape của COP có nhiều “điểm chết”.
    \end{itemize}
    \textbf{Mô tả:}
    \begin{itemize}
        \item Dùng biến \texttt{stagnation\_counter} đếm số vòng không cải thiện best.
        \item Nếu $\ge$ ngưỡng: thay $\sim$20\% cá thể tệ nhất bằng nghiệm seed mới.
    \end{itemize}

    \textbf{Khắc phục nhược điểm:}
    \begin{itemize}
        \item Giữ đa dạng quần thể.
        \item Cung cấp hướng tìm mới khi bị kẹt.
        \item Kết hợp propagation định hướng để tìm nghiệm tốt hơn trong thời gian giới hạn.
    \end{itemize}
\end{frame}
